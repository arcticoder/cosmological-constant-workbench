\documentclass[11pt]{article}

\usepackage[margin=1in]{geometry}
\usepackage{amsmath,amssymb}
\usepackage{hyperref}
\usepackage{graphicx}
\usepackage{booktabs}
\usepackage{pgfplots}
\pgfplotsset{compat=1.18}

\title{Constraints on Loop Quantum Gravity Predictions for the Cosmological Constant\\
from Phenomenological Integration and Bounded Parameter Scans}
\author{(Work-in-progress manuscript)}
\date{January 2026}

\begin{document}
\maketitle

\begin{abstract}
We present a falsifiability-oriented workflow for confronting loop-quantum-gravity (LQG)-inspired vacuum-energy estimates with late-time cosmological targets.
Using a modular ``cosmological-constant workbench'' and a lightweight integration layer, we incorporate an external LQG cosmological-constant predictor as a target generator and evaluate its implied vacuum energy density $\rho_{\rm vac}$ against the observed dark-energy density $\rho_{\Lambda,0}$.
We then perform bounded scans over a small, natural-parameter grid (108 evaluations) covering polymer and phenomenological coefficients.
Across this bounded domain, we find a persistent mismatch of at least $\sim 150$ orders of magnitude in $\rho_{\rm vac}/\rho_{\Lambda,0}$.
Within this regime, the predictor does not naturally resolve the cosmological-constant hierarchy without additional tuning or new ingredients.
The scan therefore provides an empirical no-go style constraint on the current formulation of this class of LQG-inspired vacuum-energy calculations.
\end{abstract}

\section{Motivation}
The cosmological-constant problem is often framed as a hierarchy between naive vacuum-energy scales and the observed late-time dark-energy density.
A useful contribution short of a constructive solution is a rigorous bound (or no-go statement) that excludes broad parameter regions or entire formulations.

This manuscript documents such a bound for a specific LQG-inspired vacuum-energy predictor, obtained via a reproducible integration and bounded parameter scan.

\section{Workbench and Target Integration}
We use a modular Python ``workbench'' that:
(i) defines cosmological targets and comparisons in consistent units,
(ii) supports optional external integrations behind a guarded adapter layer, and
(iii) provides constrained ``phenomenological'' mechanisms (e.g., holographic dark energy, sequestering) which can be tuned to a target $\rho_{\Lambda}$.

For the hybrid test in this work, the external LQG predictor is used solely to provide the target vacuum density (and/or an effective $\Lambda_{\rm eff}$). The question is whether the predictor is naturally near the observed scale.

\section{Bounded Parameter Scan}
We evaluate the external predictor at a fixed reference target scale (here $\ell = 10^{-15}\,\mathrm{m}$) and scan a bounded grid over four parameters:
\begin{itemize}
\item polymer parameter $\mu_{\rm polymer} \in \{0.05, 0.10, 0.15, 0.20\}$,
\item coefficient $\gamma_{\rm coefficient} \in \{0.3, 1.0, 3.0\}$,
\item scaling $\alpha_{\rm scaling} \in \{0.05, 0.10, 0.20\}$,
\item volume cutoff $j_{\max} \in \{5, 10, 20\}$.
\end{itemize}
This yields $4\times 3\times 3\times 3 = 108$ evaluations.
For each evaluation we compute
\begin{equation}
\Delta \equiv \log_{10}\left(\frac{\rho_{\rm pred}}{\rho_{\Lambda,0}}\right),
\end{equation}
where $\rho_{\Lambda,0}$ is computed from a standard background ($H_0 = 67.4\,\mathrm{km\,s^{-1}\,Mpc^{-1}}$, $\Omega_{\Lambda}=0.6889$).

\section{Results: A 150+ Order Mismatch}
Figure~\ref{fig:hist} shows the distribution of $\Delta$ over the scan grid.
In the bounded domain we find
\begin{equation}
\min\limits_{\rm scan}\, |\Delta| \approx 150.7,
\end{equation}
so the smallest mismatch over the scan is still $\mathcal{O}(10^{150})$ in $\rho_{\rm pred}/\rho_{\Lambda,0}$.

\begin{figure}[t]
\centering
\begin{tikzpicture}
\begin{axis}[
    width=0.95\linewidth,
    height=0.45\linewidth,
    xlabel={$\Delta = \log_{10}(\rho_{\rm pred}/\rho_{\Lambda,0})$},
    ylabel={Count},
    ymajorgrids=true,
    xmajorgrids=true,
]
\addplot+[
    hist={bins=18, data min=140, data max=160},
    fill=black!20,
    draw=black
] table [y=log10_ratio] {data/scan_results.tsv};
\end{axis}
\end{tikzpicture}
\caption{Histogram of $\Delta = \log_{10}(\rho_{\rm pred}/\rho_{\Lambda,0})$ over a bounded 108-point scan.
Within this scan, the predictor remains $\gtrsim 150$ orders of magnitude above the observed dark-energy density.}
\label{fig:hist}
\end{figure}

\section{Interpretation and Scope}
This result should be interpreted as an empirical no-go statement for the predictor \emph{as implemented} and \emph{within the bounded scan domain}.
It does not rule out all LQG approaches to $\Lambda$.
Rather, it rules out the claim that the present formulation (polymer + phenomenological coefficients in the scanned range) naturally lands near the observed scale.

Possible escape routes include:
(i) a qualitatively different amplitude evaluation or suppression mechanism,
(ii) a revised unit/normalization convention that changes the mapping to physical $\rho_{\rm vac}$,
or (iii) additional dynamics (e.g., true sequestering-like cancellations) not present in the predictor.

\section{Reproducibility}
The scan is fully reproducible from the repository.
To regenerate the scan data file used by Figure~\ref{fig:hist}:
\begin{verbatim}
PYTHONPATH=src python papers/lqg_cc_constraints/generate_scan_data.py
\end{verbatim}
This writes \texttt{papers/lqg\_cc\_constraints/data/scan\_results.tsv}.

To build the PDF (regenerates data and runs \texttt{latexmk} / \texttt{pdflatex}):
\begin{verbatim}
bash papers/lqg_cc_constraints/build_paper.sh
\end{verbatim}

\section{Conclusion}
Within a bounded, natural-parameter scan, the integrated LQG-inspired predictor yields vacuum energy densities that exceed the observed dark-energy density by at least $\sim 150$ orders of magnitude.
This constitutes a stringent empirical constraint on the current formulation and motivates either new ingredients or an alternative route to spin-foam amplitude suppression if LQG is to address the cosmological-constant hierarchy.

\end{document}
